% Options for packages loaded elsewhere
\PassOptionsToPackage{unicode}{hyperref}
\PassOptionsToPackage{hyphens}{url}
\PassOptionsToPackage{dvipsnames,svgnames,x11names}{xcolor}
%
\documentclass[
  letterpaper,
  DIV=11,
  numbers=noendperiod]{scrartcl}

\usepackage{amsmath,amssymb}
\usepackage{iftex}
\ifPDFTeX
  \usepackage[T1]{fontenc}
  \usepackage[utf8]{inputenc}
  \usepackage{textcomp} % provide euro and other symbols
\else % if luatex or xetex
  \usepackage{unicode-math}
  \defaultfontfeatures{Scale=MatchLowercase}
  \defaultfontfeatures[\rmfamily]{Ligatures=TeX,Scale=1}
\fi
\usepackage{lmodern}
\ifPDFTeX\else  
    % xetex/luatex font selection
\fi
% Use upquote if available, for straight quotes in verbatim environments
\IfFileExists{upquote.sty}{\usepackage{upquote}}{}
\IfFileExists{microtype.sty}{% use microtype if available
  \usepackage[]{microtype}
  \UseMicrotypeSet[protrusion]{basicmath} % disable protrusion for tt fonts
}{}
\makeatletter
\@ifundefined{KOMAClassName}{% if non-KOMA class
  \IfFileExists{parskip.sty}{%
    \usepackage{parskip}
  }{% else
    \setlength{\parindent}{0pt}
    \setlength{\parskip}{6pt plus 2pt minus 1pt}}
}{% if KOMA class
  \KOMAoptions{parskip=half}}
\makeatother
\usepackage{xcolor}
\setlength{\emergencystretch}{3em} % prevent overfull lines
\setcounter{secnumdepth}{-\maxdimen} % remove section numbering
% Make \paragraph and \subparagraph free-standing
\makeatletter
\ifx\paragraph\undefined\else
  \let\oldparagraph\paragraph
  \renewcommand{\paragraph}{
    \@ifstar
      \xxxParagraphStar
      \xxxParagraphNoStar
  }
  \newcommand{\xxxParagraphStar}[1]{\oldparagraph*{#1}\mbox{}}
  \newcommand{\xxxParagraphNoStar}[1]{\oldparagraph{#1}\mbox{}}
\fi
\ifx\subparagraph\undefined\else
  \let\oldsubparagraph\subparagraph
  \renewcommand{\subparagraph}{
    \@ifstar
      \xxxSubParagraphStar
      \xxxSubParagraphNoStar
  }
  \newcommand{\xxxSubParagraphStar}[1]{\oldsubparagraph*{#1}\mbox{}}
  \newcommand{\xxxSubParagraphNoStar}[1]{\oldsubparagraph{#1}\mbox{}}
\fi
\makeatother


\providecommand{\tightlist}{%
  \setlength{\itemsep}{0pt}\setlength{\parskip}{0pt}}\usepackage{longtable,booktabs,array}
\usepackage{calc} % for calculating minipage widths
% Correct order of tables after \paragraph or \subparagraph
\usepackage{etoolbox}
\makeatletter
\patchcmd\longtable{\par}{\if@noskipsec\mbox{}\fi\par}{}{}
\makeatother
% Allow footnotes in longtable head/foot
\IfFileExists{footnotehyper.sty}{\usepackage{footnotehyper}}{\usepackage{footnote}}
\makesavenoteenv{longtable}
\usepackage{graphicx}
\makeatletter
\newsavebox\pandoc@box
\newcommand*\pandocbounded[1]{% scales image to fit in text height/width
  \sbox\pandoc@box{#1}%
  \Gscale@div\@tempa{\textheight}{\dimexpr\ht\pandoc@box+\dp\pandoc@box\relax}%
  \Gscale@div\@tempb{\linewidth}{\wd\pandoc@box}%
  \ifdim\@tempb\p@<\@tempa\p@\let\@tempa\@tempb\fi% select the smaller of both
  \ifdim\@tempa\p@<\p@\scalebox{\@tempa}{\usebox\pandoc@box}%
  \else\usebox{\pandoc@box}%
  \fi%
}
% Set default figure placement to htbp
\def\fps@figure{htbp}
\makeatother

\KOMAoption{captions}{tableheading}
\makeatletter
\@ifpackageloaded{caption}{}{\usepackage{caption}}
\AtBeginDocument{%
\ifdefined\contentsname
  \renewcommand*\contentsname{Table of contents}
\else
  \newcommand\contentsname{Table of contents}
\fi
\ifdefined\listfigurename
  \renewcommand*\listfigurename{List of Figures}
\else
  \newcommand\listfigurename{List of Figures}
\fi
\ifdefined\listtablename
  \renewcommand*\listtablename{List of Tables}
\else
  \newcommand\listtablename{List of Tables}
\fi
\ifdefined\figurename
  \renewcommand*\figurename{Figure}
\else
  \newcommand\figurename{Figure}
\fi
\ifdefined\tablename
  \renewcommand*\tablename{Table}
\else
  \newcommand\tablename{Table}
\fi
}
\@ifpackageloaded{float}{}{\usepackage{float}}
\floatstyle{ruled}
\@ifundefined{c@chapter}{\newfloat{codelisting}{h}{lop}}{\newfloat{codelisting}{h}{lop}[chapter]}
\floatname{codelisting}{Listing}
\newcommand*\listoflistings{\listof{codelisting}{List of Listings}}
\makeatother
\makeatletter
\makeatother
\makeatletter
\@ifpackageloaded{caption}{}{\usepackage{caption}}
\@ifpackageloaded{subcaption}{}{\usepackage{subcaption}}
\makeatother

\usepackage{bookmark}

\IfFileExists{xurl.sty}{\usepackage{xurl}}{} % add URL line breaks if available
\urlstyle{same} % disable monospaced font for URLs
\hypersetup{
  colorlinks=true,
  linkcolor={blue},
  filecolor={Maroon},
  citecolor={Blue},
  urlcolor={Blue},
  pdfcreator={LaTeX via pandoc}}


\author{}
\date{}

\begin{document}


\paragraph{Study details}\label{study-details}

\subparagraph{Authors}\label{authors}

Sayan Kundu, Mohit Jaiswal, Venkanna Babu Mullapudi, Jiatong Guo, Manasi
Kamat, Kari B. Basso, Zhongwu Guo

\subparagraph{Journal}\label{journal}

Chemistry - A European Journal
\url{https://doi.org/10.1002/chem.202303047}

\subparagraph{Abstract}\label{abstract}

Glycosylphosphatidylinositols (GPIs) need to interact with other
components in the cell membrane to transduce transmembrane signals. A
bifunctional GPI probe was employed for photoaffinity‐based proximity
labelling and identification of GPI‐interacting proteins in the cell
membrane. This probe contained the entire core structure of GPIs and was
functionalized with photoreactive diazirine and clickable alkyne to
facilitate its crosslinking with proteins and attachment of an affinity
tag. It was disclosed that this probe was more selective than our
previously reported probe containing only a part structure of the GPI
core for cell membrane incorporation and an improved probe for studying
GPI‐cell membrane interaction. Eighty‐eight unique membrane proteins,
many of which are related to GPIs/GPI‐anchored proteins, were identified
utilizing this probe. The proteomics dataset is a valuable resource for
further analyses and data mining to find new GPI‐related proteins and
signalling pathways. A comparison of these results with those of our
previous probe provided direct evidence for the profound impact of GPI
glycan structure on its interaction with the cell membrane.

\subparagraph{Lipid probes utilized}\label{lipid-probes-utilized}

Bifunctional Glycosylphosphatidylinositol (bf-GPI)

Bifunctional phosphatidyl moiety with \alpha\&-D-glucoside (Control)

\subparagraph{Cell line analyzed}\label{cell-line-analyzed}

\href{https://www.atcc.org/products/ccl-2}{HeLa}

\paragraph{Uncaging \& Crosslinking
timeline}\label{uncaging-crosslinking-timeline}

\begin{longtable}[]{@{}
  >{\raggedright\arraybackslash}p{(\linewidth - 8\tabcolsep) * \real{0.2000}}
  >{\centering\arraybackslash}p{(\linewidth - 8\tabcolsep) * \real{0.2000}}
  >{\centering\arraybackslash}p{(\linewidth - 8\tabcolsep) * \real{0.2000}}
  >{\centering\arraybackslash}p{(\linewidth - 8\tabcolsep) * \real{0.2000}}
  >{\centering\arraybackslash}p{(\linewidth - 8\tabcolsep) * \real{0.2000}}@{}}
\toprule\noalign{}
\begin{minipage}[b]{\linewidth}\raggedright
Lipid Probe
\end{minipage} & \begin{minipage}[b]{\linewidth}\centering
Uptake time
\end{minipage} & \begin{minipage}[b]{\linewidth}\centering
Uncaging time
\end{minipage} & \begin{minipage}[b]{\linewidth}\centering
Interaction time
\end{minipage} & \begin{minipage}[b]{\linewidth}\centering
Crosslinking time
\end{minipage} \\
\midrule\noalign{}
\endhead
\bottomrule\noalign{}
\endlastfoot
bf-GPI & 180 min & NA & NA & 15 min \\
Control & 180 min & NA & NA & 15 min \\
\end{longtable}

\subparagraph{Mass spectrometry quantification
method}\label{mass-spectrometry-quantification-method}

Precursor ion intensity Label Free Quantitation (LFQ), performed using
Proteome Discoverer

\subparagraph{\texorpdfstring{Additional sample preparation
{?}}{Additional sample preparation ?}}\label{additional-sample-preparation}

\paragraph{Data wrangling}\label{data-wrangling}

Use the download button below to download the R script used to wrangle
the authors' original submission into the data visualization tools on
the Lipid Interactome:

Use the download button below to download the original data file prior
to wrangling:




\end{document}
